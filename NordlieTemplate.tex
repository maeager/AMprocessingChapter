{%\vspace{2ex}
\small\linespread{0.5}
\begin{table}[ptb]
    \centering
    \caption{CNSM Model Summary for AM tones}\label{tab:AMModelSummary}
%  \end{table}
%\noindent%
\begin{tabularx}{\textwidth}{|l|X|}\hline %
\hdr{2}{i}{Model Summary}\\\hline
         \textbf{Populations}          & HSR and LSR ANFs, GLG, DS, TV and three TS cells \\\hline
          \textbf{Topology}            & Tono-topicity of the cat AN and CN \\\hline
        \textbf{Connectivity}          & ANF$\to${Golgi,DS,TV,TS}, Golgi$\to$DS,TS, DS$\to${TV,TS}, and TV$\to$TS  \\\hline
         \textbf{Input model}          & ANF~model: instantaneous-rate Poisson model \citep{ZilanyBruce:2007} \\\hline
\multirow{4}{*}{\textbf{Neuron model}} & GLG cell: instantaneous-rate Poisson model\\
                                       & DS cell: HH-like single-compartment Type I-II RM model \citep{RothmanManis:2003b}\\ 
                                       & TV cell:  HH-like single-compartment Type I-c RM model \citep{RothmanManis:2003b}\\
                                       & TS cell: HH-like single-compartment Type I-t RM model \citep{RothmanManis:2003b}\\ \hline
       \textbf{Channel models}         & $I_{\textrm{Na}}$, $I_{\textrm{KHT}}$, $I_{\textrm{KLT}}$, $I_{\textrm{KA}}$ and $I_{\textrm{h}}$ \citep{RothmanManis:2003b}\\\hline
        \textbf{Synapse model}         & AMPA (\textit{ExpSyn}), GABA$_{\rm A}$ (\textit{Exp2Syn}), Glycine (\textit{Exp2Syn}) \\\hline
            \textbf{Input}             & Amplitude modulated tones, $f_c=8.91$ Hz ($f_m$ and SPL varied according to stimulus paradigm)\\\hline
        \textbf{Measurements}          & Spikes recorded over 50 repetitions.  Mean rate and SI measured from the spike times 20 ms after stimulus onset. \\\hline
\end{tabularx}

\vspace{1ex}
% - B -----------------------------------------------------------------------------
\begin{tabularx}{\textwidth}{|l|X|X|}\hline
\hdr{3}{ii}{Populations}\\\hline
\textbf{Name} &            \textbf{Elements}            & \textbf{Size} \\\hline
     HSR      &  ANF model (SR=50 Hz)   & $N_{\text{HSR}} = 50$ per freq.\ channel \\\hline
     LSR      &  ANF model (SR=0.1 Hz)  & $N_{\text{LSR}}= 20$  per freq.\ channel \\\hline
     GLG      &  GLG neural model   & $N_{\text{GLG}}= 1$  per freq.\ channel  \\\hline
     DS       &    Type I-II RM model     & $N_{\text{DS}}= 1$ per freq.\ channel \\\hline
     TV       &  Type I-classic RM model  & $N_{\text{TV}}= 1$ per freq.\ channel\\\hline
     TS$^{\ast}$ & Type I-transient RM model & $N_{\text{TS}}= 1$ per freq.\ channel\\\hline
\end{tabularx}
{\footnotesize $^{\ast}$ Three chopper subtypes were simulated at different times.}
\end{table}
\vspace{1ex}
\begin{table}[ptb]
\centering
  \caption*{CNSM Model Summary for AM tones -- continued}
% - C ------------------------------------------------------------------------------
\noindent%
\begin{tabularx}{\textwidth}{|l|l|l|X|}\hline
\hdr{4}{iii}{Connectivity}\\\hline
 \textbf{Name}   & \textbf{Source} & \textbf{Target} & \textbf{Pattern} \\\hline
    \ANFGLG      &    HSR, LSR     &       GLG       & 
Gaussian convergence, spread \sLSRGLG \sHSRGLG, weight \wHSRGLG \wLSRGLG, delay \dANFGLG \\\hline
     \ANFDS      &    HSR, LSR     &       DS        & 
Skewed Gaussian convergence,  2 octave spread below CF and 1 octave spread above CF, weight \wHSRDS \wLSRDS, number \nHSRDS \nLSRDS, delay \dANFDS \\\hline
     \ANFTV      &    LSR, HSR     &       TV        & 
Narrowband connection, weight \wLSRTV and \wHSRTV, number \nLSRTV and \nHSRTV, delay \dANFTV \\\hline
     \ANFTS      &    LSR, HSR     &       TS        & 
Narrowband connection, weight \wLSRTS and \wHSRTS, number \nLSRTS and \nHSRTS, delay \dANFTS \\\hline
     \GLGDS      &       GLG       &       DS        & 
Gaussian convergence, spread $\sGLGDS$, uniform weight \wGLGDS, number \nGLGDS, delay \dGLGDS \\\hline
     \DSTV       &       DS        &       TV        & Gaussian convergence with offset \oDSTV, with spread \sDSTV, weight \wDSTV \\\hline
     \GLGTS      &       GLG       &       TS        & 
Gaussian convergence, spread $\sGLGTS$, weight \wGLGTS, number \nGLGTS, delay $\dGLGTS=0.5$ ms \\\hline
     \DSTS       &       DS        &       TS        & 
Gaussian convergence, spread $\sDSTS$, weight \wDSTS, number \nDSTS, delay $\dDSTS=0.5$ ms \\\hline
     \TVTS       &       TV        &       TS        & 
Gaussian convergence, spread $\sTVTS$, weight \wTVTS, number \nTVTS, delay $\dTVTS=0.5$ ms \\\hline
\end{tabularx}

\vspace{1ex}
\begin{tabularx}{\textwidth}{|l|X|}\hline
\hdr{2}{iv}{Neuron and Synapse Model}\\\hline
 \textbf{Name} & DS, TV and TS cell models \\\hline
 \textbf{Type} & \INa, \IKHT, \IKLT, \IKA, \Ih, and \Ileak currents \citep{RothmanManis:2003b}, conductance synapse input \\\hline
\raisebox{-4.5ex}{\parbox{0.2\textwidth}{\textbf{Subthreshold dynamics}}}& % \\%&%
\rule{1em}{0em}\vspace*{-3.5ex}
\begin{eqnarray*}
 I_{{\rm Na}} (t,V)=\bar{g}_{{\rm Na}} m^{3} h (V-E_{{\rm Na}} )\quad,\\
 I_{KA} (t,V)=\bar{g}_{{\rm KA}} a^{4} b c (V-E_{K})\quad,\\
 I_{{\rm h}} (t,V)=\bar{g}_{{\rm h}} r (V-E_{{\rm h}} )\\
    I_{{\rm KHT}}(t,V)=\bar{g}_{{\rm KHT}} ( \frac{n^{2}+p}{2})(V-E_{K})\quad,\\
 I_{{\rm KLT}}(t,V)=\bar{g}_{{\rm KLT}} w^{3} z (V-E_{K} )\\
\end{eqnarray*} \vspace*{-5.5ex}\rule{1em}{0em}
\\\hline
 \textbf{Spiking} & Emit spike when $V(t)\geq -20$ mV  \\\hline
\end{tabularx}
\end{table}
\vspace{1ex}
\begin{table}[ptb]
  {CNSM Model Summary for AM tones -- continued}\\ 
\vspace{1ex}
\begin{tabularx}{\textwidth}{|l|X|}\hline %
\hdr{2}{v}{Input\slash Ouput}\\\hline
\multirow{3}{*}{\textbf{Input Stimulus}} & AM stimuli with fixed parameters for all paradigms: carrier frequency 8.91 kHz and 100\% modulation depth, 150 ms duration, 2 ms cosine squared on\slash off ramp, and 20 ms delay.\\ 
&F$_{0}$ response paradigm: $\fm=100$ Hz, with 5 dB \SPL increments from 0 to 90 dB SPL. \\ 
&MTF paradigm: $\fm$ varied from 50 to 1200 Hz in 50 Hz increments at sound levels 20, 40, 60 and 80 dB SPL.   \\\hline 
    \textbf{Input}      & Stimulus induced Poisson spike trains from GLG units, HSR and LSR ANFs.\\\hline
%\multicolumn{2}{|c|}{\begin{minipage}[c]{0.8\textwidth}
%\includegraphics[width=0.8\textwidth,keepaspectratio]{./gfx/Notch-Wl-12.5kHz-0.5.eps}
%\end{minipage}}\\\hline
   % \textbf{Output}    & Output of 100 TV cells, across the network, with 25 repetitions\\\hline
%\multicolumn{2}{|c|}{\begin{minipage}[c]{0.8\textwidth}%
%\includegraphics[width=0.8\textwidth,keepaspectratio]{./gfx/AN_rateplace_12.5_0.5.eps}
%\end{minipage}}\\\hline
%\textbf{Measurements}    & PSTH sampled at each click for 2 ms to measure click recovery\\\hline
%\textbf{Optimisation}    & Parameters for \GLGDS are optimised based on experimental click recovery date from \citet{BackoffPalombiEtAl:1997}. The praxis method is used for optimisation.  \\\hline
\textbf{Measurements}    &  
Mean rate and synchronisation index were calculated from each AN and CN unit for each stimuli over 50 repetitions.
% Synchronisation index was calculated relative to the modulation frequency using the vector strength method \citep{KajikawaHackett:2005}.
\\\hline
\end{tabularx}
\end{table}
}

%  \textbf{Assumptions}    & The spread ANF to DS cells (\sANFDSh,\sANFDSl) is arbitrary at this point and will be explored in the next experiment.\\ \hline
%   \textbf{Function}     & Weighted mean squared error see listing below  \\ \hline




%%% Local Variables: 
%%% mode: latex
%%% TeX-PDF-mode: nil
%%% TeX-master: "Eager_ISSNIP2011"
%%% End: 
