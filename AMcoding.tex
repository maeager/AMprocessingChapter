
\section{Amplitude Modulation in the Stellate Microcircuit}

\textit{Extracted from ISSNIP Conference paper}

\subsection*{Outline}
This paper presents the amplitude modulated coding behaviour in a
biophysically-realistic neural network model of the cochlear nucleus. It focuses
specifically on the microcircuit regulating the output of T~stellate (TS) cells.
TS cells provide a robust spectral representation of auditory information, and
enhanced coding of temporal information (especially the pitch in complex
environments \cite{KeilsonRichardsEtAl:1997}).  The model contains three
inhibitory interneurons, each controlling the rate and temporal response of TS
cells.  Despite saturation of TS units on their characteristic frequency, their
ability to code temporal information is not disabled.  The optimal coding of
amplitude modulation is still provided across the network, more specifically in
lateral sidebands around the carrier frequency.



\subsection{Introduction}
%

This paper simulates the amplitude modulation (AM) coding behaviour
in a biophysically-realistic neural network model of the CN
stellate microcircuit. TS cells are the major output of the CN and
provide a robust spectral representation of auditory information.
TS cells also show enhanced coding of temporal information,
particularly in the synchronisation to modulation in speech
sounds~\cite{BlackburnSachs:1990,KeilsonRichardsEtAl:1997}.  The
coding of AM in neurons is measured using a modulation transfer
function (MTF), which is calculated using its firing rate (rMTF) or
temporal information (tMTF).

To develop and optimise detailed neural models and neural network
models, reproducible research methods are required. The Nordlie
approach to reproducible neural network simulations
\cite{NordlieGewaltigEtAl:2009} is followed in
Table~\ref{tab:TSModelSummary}. 
% The optimisation of the parameters
% in the CN stellate model is the subject of the doctoral research of
% Michael Eager.


\begin{figure}[tb]
  \centerline{\includegraphics[width=0.9\columnwidth]{CNcircuit}}
  \caption{Cochlear nucleus stellate microcircuit. Each cell type
    is shown with its response area (frequency (F) vs sound level
    (L)) and peri-stimulus time histogram (PSTH).  Synapse types:
    Excitatory (open triangle), glycinergic (closed circle), and
    GABAergic (closed rectangle).} \label{fig:microcircuit}
\end{figure}

\subsection{Cochlear Nucleus Stellate Microcircuit}

The tonotopic organisation of the auditory pathway (i.e.\ the continuous
mapping of sound frequency to place of resonance in the cochlea) is
transferred to the CN through the population of auditory nerve fibres
(ANFs) \cite{Lorente:1981}.


The CN stellate network model drawn in Figure~\ref{fig:microcircuit}
describes the following cells and models:
\begin{enumerate}
\item The auditory nerve fibre model: The base line in
  Figure~\ref{fig:microcircuit} is a simplification of ANFs from
  low characteristic frequency (CF) to high CF (left to right).
  The model reproduces responses for high and low spontaneous rate
  (SR) ANFs at 100 channels across the frequency range 200 Hz to 48
  kHz.
\item The Golgi cell: A {GABA}ergic VCN~marginal shell unit is
  assumed to regulate excitability in the GCD~and core VCN~units
  \cite{FerragamoGoldingEtAl:1998}.  Only one \textit{in vivo}
  study has recorded extracellular data in the marginal shell area
  of the CN~\cite{GhoshalKim:1997}.  The presumed characteristics
  of Golgi cells are taken from that study and are defined by a
  monotonic response to tones and noise,
  and an unusual or chopper peri-stimulus time histogram (PSTH).
\item The D~stellate cell: A glycinergic, large multipolar cell
  with OnC PSTH~response  that acts as a coincidence detector.  Its
  large dendritic area increases its response to noise allowing it
  to behave as a wide-band inhibitor in the VCN, DCN, and
  contralateral
  CN~\cite{SmithMassieEtAl:2005,ArnottWallaceEtAl:2004,NeedhamPaolini:2007}.
\item The Tuberculoventral cell: A glycinergic, type II~{EIRA}~unit
  in the deep layer of the DCN~\cite{SpirouDavisEtAl:1999}.  This
  cell acts as a delayed echo-suppressor and narrow-band inhibitor,
  with recurrent connections between D~and T~stellate cells in the
  VCN~\cite{Alibardi:2006,OertelWickesberg:1993,WickesbergWhitlonEtAl:1991}.
\item The T~stellate cell: One of the major output projection cells
  of the CN to the inferior colliculus.  This
  multipolar neuron has been shown to have robust spectral
  representation and enhanced synchronisation to modulation in
  speech
  sounds~\cite{BlackburnSachs:1990,KeilsonRichardsEtAl:1997}.
\end{enumerate}

%%{%\vspace{2ex}
%\small\linespread{0.5}
\begin{table*}[ptb]
    \caption{CNSM Model Summary for AM tones}\label{tab:AMModelSummary}
%  \end{table}
%\noindent%
\begin{tabularx}{\textwidth}{|l|X|}\hline %
\hdr{2}{i}{Model Summary}\\\hline
         \textbf{Populations}          & HSR \& LSR ANFs, Golgi, DS, TV and three TS cells \\\hline
          \textbf{Topology}            & Tono-topicity of the cat AN and CN \\\hline
        \textbf{Connectivity}          & ANF$\to${Golgi,DS,TV,TS}, Golgi$\to$DS,TS, DS$\to$TV,TS, and TV$\to$TS  \\\hline
         \textbf{Input model}          & ANF~model: instantaneous-rate Poisson model \cite{ZilanyBruce:2007} \\\hline
\multirow{4}{*}{\textbf{Neuron model}} & Golgi: instantaneous-rate Poisson model\\
                                       & D~stellate cell: HH-like single-compartment type I-II R\&M model \cite{RothmanManis:2003b}\\ 
                                       & Tuberculoventral cell:  HH-like single-compartment type I-c R\&M model \cite{RothmanManis:2003b}\\
                                       & T~stellate cell: HH-like single-compartment type I-t R\&M model \cite{RothmanManis:2003b}\\ \hline
       \textbf{Channel models}         & $I_{\textrm{Na}}$, $I_{\textrm{KHT}}$, $I_{\textrm{KLT}}$, $I_{\textrm{KA}}$ and $I_{\textrm{h}}$ \cite{RothmanManis:2003b}\\\hline
        \textbf{Synapse model}         & AMPA (\textit{ExpSyn}), GABA$_{\rm A}$ (\textit{Exp2Syn}), Glycine (\textit{Exp2Syn}) \\\hline
            \textbf{Input}             & Amplitude modulated tones, $f_c=8.91$ Hz ($f_m$ and SPL varied according to stimulus paradigm)\\\hline
        \textbf{Measurements}          & Spikes recorded over 50 repetitions.  Mean rate and synchronisation index measured from the spike times 20 ms after stimulus onset. \\\hline
\end{tabularx}

\vspace{1ex}
% - B -----------------------------------------------------------------------------
\begin{tabularx}{\textwidth}{|l|X|X|}\hline
\hdr{3}{ii}{Populations}\\\hline
\textbf{Name} &            \textbf{Elements}            & \textbf{Size} \\\hline
     HSR      &  Poisson model with refractory effects  & $N_{\text{HSR}} = 50$ per freq.\ channel \\\hline
     LSR      &  Poisson model with refractory effects  & $N_{\text{LSR}}= 20$  per freq.\ channel \\\hline
     GLG      &  Poisson model with refractory effects  & $N_{\text{GLG}}= 1$  per freq.\ channel  \\\hline
     DS       &    Type I-II Rothman \& Manis model     & $N_{\text{DS}}= 1$ per freq.\ channel \\\hline
     TV       &  Type I-classic Rothman \& Manis model  & $N_{\text{TV}}= 1$ per freq.\ channel\\\hline
     TS       & Type I-transient Rothman \& Manis model & $N_{\text{TS}}= 1$ per freq.\ channel\\\hline
\end{tabularx}

\vspace{1ex}
% - C ------------------------------------------------------------------------------
%\noindent%
\begin{tabularx}{\textwidth}{|l|l|l|X|}\hline
\hdr{4}{iii}{Connectivity}\\\hline
 \textbf{Name}   & \textbf{Source} & \textbf{Target} & \textbf{Pattern} \\\hline
    \ANFGLG      &    HSR, LSR     &       GLG       & 
Gaussian convergence, spread \sLSRGLG \sHSRGLG, weight \wHSRGLG \wLSRGLG, delay \dANFGLG \\\hline
     \ANFDS      &    HSR, LSR     &       DS        & 
Skewed Gaussian convergence,  2 octave spread below CF and 1 octave spread above CF, weight \wHSRDS \wLSRDS, number \nHSRDS \nLSRDS, delay \dANFDS \\\hline
     \ANFTV      &    LSR, HSR     &       TV        & 
Narrowband connection, weight \wLSRTV and \wHSRTV, number \nLSRTV and \nHSRTV, delay \dANFTV \\\hline
     \ANFTS      &    LSR, HSR     &       TS        & 
Narrowband connection, weight \wLSRTS and \wHSRTS, number \nLSRTS and \nHSRTS, delay \dANFTS \\\hline
     \GLGDS      &       GLG       &       DS        & 
Gaussian convergence, spread $\sGLGDS$, uniform weight \wGLGDS, number \nGLGDS, delay \dGLGDS \\\hline
     \DSTV       &       DS        &       TV        & Gaussian convergence with offset \oDSTV, with spread \sDSTV, weight \wDSTV \\\hline
     \GLGTS      &       GLG       &       TS        & 
Gaussian convergence, spread $\sGLGTS$, weight \wGLGTS, number \nGLGTS, delay $\dGLGTS=0.5$ ms \\\hline
     \DSTS       &       DS        &       TS        & 
Gaussian convergence, spread $\sDSTS$, weight \wDSTS, number \nDSTS, delay $\dDSTS=0.5$ ms \\\hline
     \TVTS       &       TV        &       TS        & 
Gaussian convergence, spread $\sTVTS$, weight \wTVTS, number \nTVTS, delay $\dTVTS=0.5$ ms \\\hline
\end{tabularx}

\vspace{1ex}
\begin{tabularx}{\textwidth}{|l|X|}\hline
\hdr{2}{iv}{Neuron and Synapse Model}\\\hline
 \textbf{Name} & DS, TV and TS cell models \\\hline
 \textbf{Type} & Na, KA, KHT, Ih, and leak currents \cite{RothmanManis:2003b}, conductance synapse input \\\hline
\raisebox{-4.5ex}{\parbox{0.2\textwidth}{\textbf{Subthreshold dynamics}}}& % \\%&%
\rule{1em}{0em}\vspace*{-3.5ex}
\begin{eqnarray*}
 I_{{\rm Na}} (t,V)=\bar{g}_{{\rm Na}} m^{3} h (V-E_{{\rm Na}} )\quad,\quad I_{KA} (t,V)=\bar{g}_{{\rm KA}} a^{4} b c (V-E_{K})\quad,\quad I_{{\rm h}} (t,V)=\bar{g}_{{\rm h}} r (V-E_{{\rm h}} )\\
    I_{{\rm KHT}}(t,V)=\bar{g}_{{\rm KHT}} ( \frac{n^{2}+p}{2})(V-E_{K})\quad,\quad I_{{\rm KLT}}(t,V)=\bar{g}_{{\rm KLT}} w^{3} z (V-E_{K} )\\
\end{eqnarray*} \vspace*{-5.5ex}\rule{1em}{0em}
\\\hline
 \textbf{Spiking} & Emit spike when $V(t)\geq -20$ mV  \\\hline
\end{tabularx}

\vspace{1ex}
\begin{tabularx}{\textwidth}{|l|X|}\hline %
\hdr{2}{v}{Input\slash Ouput}\\\hline
\textbf{Input Stimulus} & AM stimuli with fixed parameters for all paradigms: carrier frequency 8.91 kHz and 100\% modulation depth, 150 ms duration, 2 ms cosine squared on\slash off ramp, and 20 ms delay. F$_{0}$ response paradigm: modulation frequency 100 Hz, with 5 dB SPL increments from 0 to 90 dB SPL. MTF paradigm:  \\\hline 
    \textbf{Input}      & Stimulus induced Poisson spike trains from GLG units, HSR and LSR ANFs, and natural synaptic input from DS and TV units\\\hline
%\multicolumn{2}{|c|}{\begin{minipage}[c]{0.8\textwidth}
%\includegraphics[width=0.8\textwidth,keepaspectratio]{./gfx/Notch-Wl-12.5kHz-0.5.eps}
%\end{minipage}}\\\hline
   % \textbf{Output}    & Output of 100 TV cells, across the network, with 25 repetitions\\\hline
%\multicolumn{2}{|c|}{\begin{minipage}[c]{0.8\textwidth}%
%\includegraphics[width=0.8\textwidth,keepaspectratio]{./gfx/AN_rateplace_12.5_0.5.eps}
%\end{minipage}}\\\hline
%\textbf{Measurements}    & PSTH sampled at each click for 2 ms to measure click recovery\\\hline
%\textbf{Optimisation}    & Parameters for \GLGDS are optimised based on experimental click recovery date from \citet{BackoffPalombiEtAl:1997}. The praxis method is used for optimisation.  \\\hline
\textbf{Measurements}    &  
Mean rate and synchronisation index were calculated from each AN and CN unit for each stimuli over 50 repetitions.
% Synchronisation index was calculated relative to the modulation frequency using the vector strength method \cite{KajikawaHackett:2005}.
\\\hline
\end{tabularx}
\end{table*}
%}

%  \textbf{Assumptions}    & The spread ANF to DS cells (\sANFDSh,\sANFDSl) is arbitrary at this point and will be explored in the next experiment.\\ \hline
%   \textbf{Function}     & Weighted mean squared error see listing below  \\ \hline




%%% Local Variables: 
%%% mode: latex
%%% TeX-PDF-mode: nil
%%% TeX-master: "Eager_ISSNIP2011"
%%% End: 


\subsection{Methods}

Tables~\ref{tab:TSModelSummary}i through \ref{tab:TSModelSummary}v
show the detailed summary of the CN stellate microcircuit used in
the AM simulations.
% The Nordlie
% \cite{NordlieGewaltigEtAl:2009} table format splits the tables into Model
% Summary, Populations, Connectivity, Neuron Model, and Input\slash
% Output.

\subsubsection{Simulations}

The simulations were performed using the neural simulation package
NEURON \cite{CarnevaleHines:2006}. NEURON's Crank-Nicholson
integration method was used in the simulations with time step, $dt=
0.05$ ms, and total duration 270 ms. Stimuli were repeated 25 times
and the spike times of all CN neurons recorded.

\subsubsection{Stimulus}

Stimulus generation follows Rhode and Greenberg's
\cite{RhodeGreenberg:1994} encoding of amplitude modulation in
cats.  AM signals were generated by modulating a carrier signal of
specified frequency, amplitude, and phase by a sinusoidal modulator
whose amplitude and phase were equal to that of the carrier (100\%
AM). The carrier frequency was set to the central frequency channel
of the CN model (5.82 kHz).  Modulation frequency, $f_m$, ranged
from 50 to 800 Hz and was stepped up by 50 Hz.

\subsubsection{Auditory Model}

The input auditory model used in this paper provides the major
phenomenological qualities of experimentally recorded ANFs. The
Zilany model \cite{ZilanyBruceEtAl:2009} is based on many auditory
models from the Carney Lab
\cite{HeinzColburnEtAl:2001,ZhangCarney:2001,Carney:1993}. The
centre frequencies for 100 channels is determined by the logarithmic
Greenwood function \cite{Greenwood:1990} of the basilar membrane in
cats. The model reproduces responses for 50 high and 30 low SR ANFs
in each frequency channel, across the frequency range 200 Hz to 64
kHz.


\subsubsection{Golgi Cell Model}

Inputs to Golgi cells are more complicated than the inputs to core
VCN~neurons.  Golgi cells are sparse in the region surrounding the
VCN called the granule cell domain.  Extracellular recordings from
labelled Golgi cells are not available in the literature; however,
the GCD~(or marginal shell of the VCN~in cats) has been studied by
one group \cite{GhoshalKim:1997} without direct labelling of
recorded units.  Any extracellular spikes recorded in the GCD~are
most likely from Golgi cells since granule cell somata are less
than $10 \mu{m}$ and their narrow axons are unlikely to elicit
electrical activity in the electrodes.  The majority of recorded
units showed a monotonic increase in firing rate with increasing
sound intensity \cite{GhoshalKim:1997}.

The Golgi cell model is implemented as an instantaneous-rate
Poisson rate model.  The primary inputs are from the auditory
model's instantaneous rate outputs with connections across
frequency channels.  HSR~and LSR~ANF~inputs to Golgi cells were
specified  by a Gaussian distribution in fibres across the network.
The weighted sum of HSR~and LSR~instantaneous-rate vectors were
smoothed out by an alpha function mimicking a synaptic and
dendritic smoothing filter.

\subsubsection{Neural Models}

The spiking neural models used in the auditory nerve fibres and
Golgi cell model are inhomogeneous Poisson processes.  The
instantaneous rate is passed through the Jackson spiking model,
which includes refractory effects typical of the auditory nerve
fibres \cite{Jackson:2003,JacksonCarney:2005}.  Spike trains for
each neuron in the model are created at the start of each
repetition of the stimulus, but can be saved and loaded from a file.


Membrane current models (Table~\ref{tab:TSModelSummary}iv) used in
DS, TV and TS cell models were developed from kinectic analysis of
ventral cochlear nucleus (VCN) neurons in mice
\cite{RothmanManis:2003b}. Their activation and deactivation
functions (\textit{a, b, c, h, m, n, p, r, w} and \textit{z}) are
described in detail by Rothman and Manis \cite{RothmanManis:2003}
and the NEURON source code is freely available online at ModelDB
\cite{HinesMorseEtAl:2004}.  Table~\ref{tab:2} shows the membrane
conductance parameters of the cell types.  
%Conductance parameters
%were adjusted from \cite{RothmanManis:2003b} due to temperature and
%soma diameter changes.
% Rothman and Manis used 22$^\circ$C slice preparation.
% Temperature effects the activation and deactivation functions'
% time constants of the current models that used 37$^\circ$C. The
% temperature quotient, Q=Q$_{10}^{((37^\circ -22^\circ )/10)}$,
% was used to adjust the current models where Q$_{10}=3.0$
% \\ %\hline
The reversal potential for potassium, sodium, leak, and h currents were -72,
0, -65, and -43 mV, respectively.

\begin{table}[!th]
  \centering
  \caption{Cell-type Membrane Current Parameters}\label{tab:2}
  \begin{tabular}{lccc}
\hline
                Cells                 &   TS   &   DS   &   TV     \\ %\hline
         Current Clamp Model          &  I-t   &  I-II  &   I-c    \\
\hline
 $\bar{g}_{{\rm Na}} $, S cm$^{-2}$   & 0.235  & 0.235  &  0.235      \\ %\hline
 $\bar{g}_{{\rm KHT}} $, S cm$^{-2}$  & 0.018  &  0.02  &  0.019      \\ %\hline
$\bar{g}_{{\rm  KLT}} $, S cm$^{-2}$  &   0    & 0.0047 &    0         \\ %\hline
 $\bar{g}_{{\rm KA}} $, S cm$^{-2}$   & 0.0153 &   0    &    0        \\ %\hline
 $\bar{g}_{{\rm h}} $, mS cm$^{-2}$   & 0.0618 & 0.247  & 0.06178     \\ %\hline
$\bar{g}_{{\rm leak}} $, mS cm$^{-2}$ & 0.471  & 0.471  &  0.471      \\ %\hline
        Soma Diameter, $\mu$m         &   21   &   25   &  19.5   \\ %\hline
    Input Resistance, M$\Omega $      &  163   &   73   &   170   \\
\hline
\end{tabular}
\end{table}


\subsubsection{Synapse Parameters}

NEURON's conductance synapse models, {\it ExpSyn} and {\it
  Exp2Syn}, were used in the CN stellate microcircuit.  Single
exponential excitatory synapses ($\tau _{{\rm AMPA}}=0.36$~ ms)
model the experimental recordings in VCN neurons
\cite{GardnerTrussellEtAl:1999}.  Double exponential inhibitory
synapses are used in the network from glycinergic and GABAergice
neurons. Glycinergic synapses \cite{LeaoOleskevichEtAl:2004} ($\tau
_{{\rm Gly1}}=0.4$ ms and $\tau _{{\rm Gly2}}=2.5$ ms) and
GABA$_{\mathrm{A}}$ synapses
\cite{AwatramaniTurecekEtAl:2005}($\tau _{{\rm GABA1}}=0.7$ ms and
$\tau _{{\rm GABA2}}=9.0$ ms) were modeled from MNTB neurons in
mature guinea pigs.  Chlorine reversal potential in Glycine and
GABA$_{\mathrm{A}}$ receptors was set to -75 mV and excitatory
reversal potential was set to 0 mV.


\subsubsection{Connectivity}

The connectivity of the cell types involved in the stellate
microcircuit is shown in Figure~\ref{fig:microcircuit} and in
Table~\ref{tab:TSModelSummary}iii. Fast, glycinergic inhibition from
TV cells and DS cells (Fig.~\ref{fig:microcircuit}) is involved in modulating the firing rate and
spike interval variability in TS cells
\cite{FerragamoGoldingEtAl:1998,WickesbergOertel:1993}. TV cells
in the deep layer of the dorsal CN, provide a delayed narrowband
inhibition to TS and DS cells in the ventral CN\@.  
%The dendrites
%of DS cells cover 1/3~of the cross-frequency axis in the CN,
%contributing to this cell's wide frequency response. In turn this
%cell is responsible for altering the frequency responses in TS and
%TV cells \cite{SpirouDavisEtAl:1999}. 
DS cells are coincidence detectors and have a precisely timed onset
response that affects the temporal properties of TS cells
\cite{PaoliniClareyEtAl:2005,RhodeGreenberg:1994a} and completely
inhibit TV cell responses to loud clicks
\cite{SpirouDavisEtAl:1999}. GABAergic inhibition from Golgi cells
modulates the level of excitation necessary to reach threshold for
all CN cells
\cite{CasparyBackoffEtAl:1994,FerragamoGoldingEtAl:1998}.
%Feedback
%circuits from the olivary complex to the ventral CN are also known
%to use GABA as a neurotransmitter \cite{SaintMorestEtAl:1989},
%however this is not included in the model.




\subsubsection{Analysis}

Temporal information was measured using the synchronisation index
relative to the modulation frequency of the stimuli.  The
synchronisation index (SI) was calculated 20 ms after the onset on
the stimulus (20 ms delay) \cite{KajikawaHackett:2005}. Vector
strength and Rayleigh coefficient were also calculated to verify
the SI values using an FFT of the period histogram.  SI values
below 0.1 are considered insignificant.

%\clearpage
\subsection{Results}

The figures below show the rate and temporal responses, across the
entire network, to an AM tone with carrier frequency equal to the
central channel's characteristic frequency (5.82 kHz).
Modulation frequency ranged from 50 to 800 Hz.  Each figure shows the mean
firing rate on the left and the synchronisation index on the right.
The sound level of each stimulus was set to 40 dB SPL for the top
row and 60 dB SPL for the bottom row.  

\subsubsection{Golgi Cells}
\begin{figure}[tb]
\centering 
%\caption{GLG Rate (spks/s) and SI 60 dB}
\caption{Golgi cell rate and temporal modulation responses}\label{fig:G}
{\hspace{0.2\columnwidth}rMTF (sp/s) \hspace{0.35\columnwidth} tMTF}\\
\resizebox{0.95\columnwidth}{!}{\includegraphics{40/ratetemporal.3.eps}}\\
\resizebox{0.95\columnwidth}{!}{\includegraphics{60/ratetemporal.3.eps}}
\end{figure}
Figure~\ref{fig:G} shows the rate and temporal MTF across the whole
network to AM tone centred at channel 50. The Golgi units had very
low rates for 40 and 60 dB SPL AM tones which were limited to a
narrow range around the central channel.  The temporal response of
Golgi units was almost non-existent except for very low modulation.

\subsubsection{D~Stellate Cells}
\begin{figure}[tb]
\centering 
\caption{DS cell rate and temporal modulation responses}\label{fig:DS}
{\hspace{0.2\columnwidth}rMTF (sp/s) \hspace{0.35\columnwidth} tMTF}\\
\resizebox{0.95\columnwidth}{!}{\includegraphics{40/ratetemporal.2.eps}}\\
\resizebox{0.95\columnwidth}{!}{\includegraphics{60/ratetemporal.2.eps}}
\end{figure}
The broad range of CF inputs to DS units allow for a greater
likelihood of coincident detection and an increase in
synchronisation relative to the inputs.  The rate responses of DS
units (Fig.~\ref{fig:DS}) were wider for 40 and 60 SPL stimuli
relative to the narrow band TS units.  For 40 dB SPL stimuli, most
DS units had a band-pass rMTF. For higher SPL, a greater number of
spikes occured between 100 and 500 Hz for units above CF (band-pass
rMTF), but the rest of the active units remained stable (low-pass
rMTF). This ``rate-responder'' behaviour is similar in ideal onset
units in the VCN (octopus cells) but the cut-off of the rMTF is much
lower. The temporal responses of DS units were predominantly
band-pass, with higher SI values than ANFs.  For lower SPL, the
responses were consistent across active units with a falling cut-off
frequency with falling CF. For high SPL, the DS units were divided
along the central channel.  The DS units above the central channel
had the strongest synchronisation and cut-off frequencies near the
upper limit of the AN model.  The DS units below the central
channel had cut-off frequencies around 400 Hz, similar to TS and TV
units.

\subsubsection{Tuberculoventral Cells}
\begin{figure}[tb]
\centering 
%\caption{TV Rate (spks/s) and SI 60 dB}
\caption{TV cell rate and temporal modulation responses}\label{fig:TV}
{\hspace{0.2\columnwidth}rMTF (sp/s) \hspace{0.35\columnwidth} tMTF}\\
\resizebox{0.95\columnwidth}{!}{\includegraphics{40/ratetemporal.1.eps}}\\
\resizebox{0.95\columnwidth}{!}{\includegraphics{60/ratetemporal.1.eps}}
\end{figure}
The rate and temporal responses of TV units (Fig.~\ref{fig:TV})
showed the non-linear effects of strong inhibition from DS
units. TS and TV units received similar ANF inputs, but the
inhibition limited the activity at low sound level and then to a
narrow range at higher SPL.  The temporal responses of TV units
were similar to TS units but with lesser synchronisation and
sharper cut-off.  The outer edges of active units provided the best
temporal response with little to no temporal information at the
carrier frequency units.


\subsubsection{T~Stellate Cells}
\begin{figure}[tb]
\centering 
%\caption{TS Rate (spks/s) and SI 60 dB}
\caption{TS cell rate and temporal modulation responses}\label{fig:TS}
{\hspace{0.2\columnwidth}rMTF (sp/s) \hspace{0.35\columnwidth} tMTF}\\
\resizebox{0.95\columnwidth}{!}{\includegraphics{40/ratetemporal.0.eps}}\\
\resizebox{0.95\columnwidth}{!}{\includegraphics{60/ratetemporal.0.eps}}
\end{figure}
Figure~\ref{fig:TS} shows the final MTF response of the TS units in
the network.  The spread of excitation in TS units was narrow
around the central channel, with greater excitation above CF around
fm=300 Hz. For higher sound levels, the spread of excitation was
wider but the rate was steadier for each stimuli.  The significant
features of the temporal responses in the right of the figure are
the very poor synchronisation in the central channel and dominant
synchronous responses at the outer edge of excitation.  For 40 dB
SPL, most active units showed a band-pass MTF; however, the
dominant units above CF (channels 55 to 58) had low-pass MTFs.  For
60 dB SPL, most active units showed band-pass MTFs except for the
central units, which showed limited results or a low-pass MTF.
Outermost active units (channels 65 to 60 and 45 to 40) had the
most dominant temporal response across the TS cell population.


\subsection{Discussion}

Golgi cells are low-firing monotonic units that influence the
general excitability of DS and TS units using GABA.  The results in
Fig.~\ref{fig:G} show that the rate and temporal response to AM
tones is only dependent on the sound level. 

The rate and temporal response of TV cells was
strongly inhibited by DS units.  TV cells are thought to be
responsible for delayed inhibition or echo-supression
\cite{WickesbergOertel:1990}, but can also be involved in tuning
the temporal MTF behaviour in TS cells.

D~stellate cells have an onset chopping behaviour to tones, but can
follow the repetition of amplitude modulated tones. The entrainment
to the stimulus envelope produced band-pass rate MTFs in DS units
with a CF above $f_c$.  The temporal information at the channel
with CF=$f_c$ (Fig.~\ref{fig:DS}) was diminished by the strong
GABAergic inhibition of Golgi cells; however, the majority of active
DS units showed strong synchronisation, which suggests 
synchronous tuning in TV and TS units throughout the CN.

Intrinsic membrane properties and synaptic connections enable TS
units to be enhanced or tuned to important features of the acoustic
input \cite{PaoliniClareyEtAl:2005}. The behaviour of TS units is
influenced by all three interneurons in the stellate
microcircuit. Rate saturation of TS units on CF (Fig.~\ref{fig:TS})
does not disable their ability to encode temporal information.
Experimental data has shown TS cells generally have low-pass MTF at
low sound level and band-pass MTF for higher sound levels for AM
tones on CF \cite{RhodeGreenberg:1994}. A whole-network approach
may provide a more advantageous basis for optimal temporal coding
of AM than an approach based solely on CF.

% The inhomogeneous T~stellate cell population are classified as
% either sustained or transient choppers.


\subsection{Conclusion}

The CN stellate microcircuit provides controlled and modulated
enhancement of the output of TS cells, one of the major outputs of
the cochlear nucleus.  This paper has shown the necessity to model
detailed neural microcircuits away from basic receptive fields of
individual units.  The model has been used for detailed
optimisation \cite{EagerGraydenEtAl:2006,EagerGraydenEtAl:2006a} so
that it can be used to investigate detailed physiological
properties in the CN stellate network.



% With the acceleration of computational
% power and enhanced experimental techniques in multi-unit
% recordings are enabling more detailed neural models to be
% developed. There is much to be gained from biophysically-
% realistic modelling approaches, especially in the heavily in-
% vestigated cochlear nucleus of mammals and birds.
% This paper simulates the amplitude modulated (AM) coding
% behaviour in a biophysically-realistic neural network model
% of the cochlear nucleus stellate microcircuit



%%% Local Variables:
%%% mode: latex
%%% mode: tex-fold
%%% mode: visual-line
%%% TeX-master: "VowelProcessing"
%%% TeX-PDF-mode: nil
%%% End:
